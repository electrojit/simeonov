\chapter[Introduction]{\thechapter. Introduction}
\section{Intention of project}
The scope of this project is aimed to help understand and identify the advantages and disadvantages of the diverse modeling languages such as UML 2.0, SysML and UML-RT. In order to achieve this goal, we need a standardized software architecture like the one offered by Automotive Open System Architecture (AUTOSAR).\newline
To achieve this case study we need to model a given part from the AUTOSAR architecture, that is to be found in every automotive unit. This is the reason why we have chosen the memory stack.
\section{Goals and success criteria of project}
The project activities are going to be comparable to the base practices(BP) defined by the Software Process Improvement and Capability Determination(SPICE) model.
\begin{itemize}
\item BP1 - Software architectural design - 30.10.2009
\item BP2 - Software requirements - 06.11.2009
\item BP3 - Define interfaces - 20.11.2009
\item BP4 - Dynamic behavior - 27.11.2009
\item BP6 - Detailed design - 28.12.2009
\item BP7 - Verification criteria - 11.01.2010
\item BP8 - Verify Software design - 18.01.2010
\item BP9 - Consistency and Bilateral Traceability - 28.01.2010
\end{itemize}
The BP5 and BP10 cannot be done in this setting due to missing work products. The project is successfully completed if the results are meaningful in industrial practice.